\documentclass{beamer}
% Configurações do tema
\usetheme{Dresden}
\usecolortheme{default}

% Pacotes necessários
\usepackage[utf8]{inputenc}
\usepackage[portuguese]{babel}
\usepackage{amsmath}
\usepackage{amsfonts}
\usepackage{amssymb}
\usepackage{graphicx}
\usepackage{listings}
\usepackage{xcolor}
\usepackage[normalem]{ulem}

% Configuração para código
\lstset{
    language=C++,
    basicstyle=\ttfamily\footnotesize,
    keywordstyle=\color{blue},
    commentstyle=\color{green},
    stringstyle=\color{red},
    numberstyle=\tiny\color{gray},
    numbers=left,
    breaklines=true,
    frame=single,
    tabsize=2
}

% Informações da apresentação
\title{Tópicos Avançados em Programação Competitiva}
\subtitle{Funções Geradoras}
\author{perchuts}
\institute{UFMG}
\date{\today}

% Logo (opcional)
% \logo{\includegraphics[height=0.4cm]{logo.png}}

\begin{document}

% Slide de título
\frame{\titlepage}

% Índice
\begin{frame}
\frametitle{Sumário}
\tableofcontents
\end{frame}

%\begin{frame}
%\frametitle{Problemas!}
%\begin{itemize}
%    %\item Ex - Typical Convolution Problem (Online Convolution) % https://atcoder.jp/contests/abc315/tasks/abc315_h
%    %\item Sum (função geradora) % https://vjudge.net/problem/LibreOJ-2409#author=GPT_en - alternativa: https://yukicoder.me/problems/no/1145
%    %\item G - Prime Circuit (operações em Séries de Potência) % https://atcoder.jp/contests/abc387/tasks/abc387_g
%    \item Ex - Antichain (Affine Composition) % https://atcoder.jp/contests/abc269/tasks/abc269_h
%    \item F - Expansion Plan (operações em Séries de Potência) % https://codeforces.com/group/TFrGcBYYxs/contest/584448/problem/F
%    \item E - Competition Graph (operações em Séries de Potência + Online Convolution) % https://codeforces.com/gym/105949/problem/E -> lista
%    \item F - Pockets (operações em Séries de Potência) % https://codeforces.com/gym/103743/problem/F
%    \item H - Random Kth Max (operações em Séries de Potência) % https://atcoder.jp/contests/abc226/tasks/abc226_h
%    % \item G - Socks 3 (convolução divide and conquer) % https://atcoder.jp/contests/abc352/tasks/abc352_g -> ruim!
%    \item F - Finding the best guess (convolução centroide) % https://codeforces.com/gym/104375/problem/F
%    \item D - Mahjong (operações em Série de Potência) % https://atcoder.jp/contests/arc160/tasks/arc160_d -> lista
%    %\item D - Binomial Coefficient is Fun (operações em Série de Potência) % https://atcoder.jp/contests/arc110/tasks/arc110_d
%    \item Ex - Constrained Tree Degree (nao sei?) % https://atcoder.jp/contests/abc303/tasks/abc303_h
%\end{itemize}
%\end{frame}

% Seção 2
\section{Definições, Exemplos e Propriedades}

\begin{frame}
\frametitle{Séries de Potência Formais}
      Funções geradoras são úteis para resolvermos problemas de combinatória algebricamente.


      Dada uma sequência $(a_n)$, dizemos que sua função geradora $F(x)$ é a série 

      $$
      F(x) = \sum_{n=0}^{\infty} a_n x^n = a_0 + x a_1 + x^2 a_2 + \dots + x^n a_n + \dots
      $$

      Quase sempre estamos interessados apenas nos $\textit{coeficientes}$ da série, então não nos importa qualquer noção de convergência. Nesse contexto dizemos que as funções geradoras são Séries de Potência Formais.
\end{frame}

\begin{frame}
\frametitle{Exemplos}
      A sequência $(a_n) = (1, 1, \dots, 1, \dots)$ tem como função geradora 
      $$
      P(x) = \sum_{n=0}^{\infty} x^n
      $$
      Podemos encontrar uma expressão para $P(x)$:

      $$P(x) - x P(x) = 1 \iff P(x) = \frac{1}{1-x}
      $$


\end{frame}

\begin{frame}
\frametitle{Exemplos}
      $(a_n)$ com $a_n = n$ tem como função geradora

      $$
      Q(x) = \sum_{n=0}^{\infty} n x^n
      $$

      Note que $P'(x) = \sum_{n=0}^{\infty} (n+1) x^{n}$

      Então

      $$
      Q(x) = x P'(x) = x (\frac{1}{1-x})' = \frac{x}{(1-x)^2}
      $$

\end{frame}

\begin{frame}
\frametitle{Exemplos}
      $(f_n)$ com $f_0 = 0$, $f_1 = 1$, e $f_{n+2} = f_{n+1} + f_n$. Qual é a função geradora $F(x)$? Não é fácil calcular seus coeficientes, mas podemos encontrar uma expressão para ela.
      \begin{align*}
      &f_{n+1} + f_n = f_{n+2} \\
      &x^{n+2} f_{n+1} + x^{n+2} f_n = x^{n+2} f_{n+2} \\
      &\sum_{n=0}^{\infty} (f_n + f_{n+1}) x^{n+2} = \sum_{n=0}^{\infty} f_{n+2} x^{n+2} \\
      &x^2 F(x) + x F(x) = F(x) - x \\
      &F(x) [x^2 +  x - 1] = -x\\
      &F(x) = \frac{x}{1-x-x^2}
      \end{align*}

\end{frame}

\begin{frame}
\frametitle{Propriedades/definições úteis}
      \begin{itemize}
            \setlength\itemsep{1.2em}
            \item Convolução: $A(x)B(x) = \sum_{n=0}^{\infty} x^n \sum_{i=0}^{n} a_i b_{n-i}$ 

            \vspace{0.3em}

            Seja $P(x) = \sum_{i=0}^\infty p_n x^n$

            \item Definimos $[x^n] P(x) = p_n$

            \item $[x^n] x^m P(x) = [x^{n-m}] P(x)$

            \item $[x^n]\frac{1}{1-x} P(x) = \sum_{i=0}^n p_i$

            \item $[x^n] \frac{1}{(1-x)^k} = \binom{n+k-1}{k-1}$
            % generalizar para [x^n]P^k

            \item Defina as séries $P_0(x) := P(x)$ e $P_{k+1}(x) := xP_k'(x)$. Vale que $[x^n] P_k(x) = n^k p_n$

            %\item Condição para inversa: Existe $Q(x)$ tal que $P(x) Q(x) = 1$ se e somente se $p_0 \ne 0$. 
      \end{itemize}
\end{frame}

\begin{frame}
\frametitle{Função Geradora Exponencial}
      Considere agora a série formal

      $$
      P(x) = \sum_{n=0}^{\infty} \frac{a_n x^n}{n!}
      $$

      $P$ é a função geradora exponencial da sequência $(a_n)$ e definimos que $[x^n] P(x) = a_n$.
      Se $A, B$ são EGFs, temos que 
      $$[\frac{x^n}{n!}] A(x) B(x) = \sum_{i=0}^{n} \binom{n}{i} p_i q_{n-i}$$

      Esse fato torna as EGFs muito úteis.
\end{frame}

\begin{frame}
      \frametitle{Como implementar?}

      Vários dos algoritmos abaixo usam a NTT como subrotina. Por isso vamos trabalhar sobre $\mathbb{Z}_{998244353}$ na maior parte do tempo. É altamente recomendável usar uma primitiva de aritmética modular.

      \begin{block}{Operações comuns}
      Seja $P(x)$ um polinômio de grau $n$. É possível computar:
      \begin{itemize}
            \item $P'(x)$ e $\int P(x)$ em $O(n)$;
            \item Primeiros $n$ termos de $P^{-1}(x)$ em $O(n \log n)$;
            \item Primeiros $n$ termos de $e^{P(x)} = \sum_{i=0}^{\infty} \frac{P^i(x)}{i!}$ em $O(n \log n)$;
            \item Primeiros $n$ termos de $\ln P(x)$ em $O(n \log n)$.
      \end{itemize}
      \end{block}

      A biblioteca da UFMG implementa todas essas operações. Entender o funcionamento dos algoritmos não é necessário para a grande maioria dos problemas.


\end{frame}

\section{Problemas}
\begin{frame}

      \begin{center}
            {\LARGE \textcolor{blue}{\textbf{Problemas!}}}
      \end{center}
      
\end{frame}

\section{ARC110D}
\begin{frame}[t]
      \frametitle{AtCoder ARC110D - Binomial Coefficient is Fun}% https://atcoder.jp/contests/arc110/tasks/arc110_d
      \framesubtitle{Enunciado}
      \only<1-2>{
      Temos uma sequência $A$ de $N$ inteiros não negativos. Compute a soma de $\prod_{i=1}^{N} \binom{B_i}{A_i}$ sobre todas as sequências $B$ de $N$ inteiros não negativos cuja soma é no máximo $M$.
      } 
      \only<2>{
      \begin{block}{Restrições}
           $1 \leq N \leq 2000$ \\
           $1 \leq M \leq 10^9$ \\
           $0 \leq A_i \leq 2000$ 
      \end{block}
      }
\end{frame}
\begin{frame}[t]
      \frametitle{AtCoder ARC110D - Binomial Coefficient is Fun}% https://atcoder.jp/contests/arc110/tasks/arc110_d
      \framesubtitle{Solução}
            \only<1-2>{

            Queremos representar aquele produtório como um produto de coeficientes de polinômios. Fixado um $i$, vamos começar encontrando a função  geradora $F_i(x)$ de $(a_n)_{n \in \mathbb{N}}$ com $a_n = \binom{n}{A_i}$.

            }
            \only<2>{
            $$
                  F_i(x) = \frac{x^{A_i}}{(1-x)^{A_i+1}}
            $$
            }
            \onslide<2->{

            Seja $P(x) = \prod_{i=1}^N F_i(x)$. Note que
            $$[x^k] P(x) = \sum_{B_1+\dots+B_N = k} \prod_{i=1}^{N} \binom{B_i}{A_i}$$
            } 

            \onslide<3->{
                  Então a resposta será igual a $\sum_{i=0}^{M} [x^i] P(x) = [x^M] \frac{1}{1-x} P(x)$.
                  Seja $S = \sum A_i$. Por fim,

                  $$
                        [x^M] \frac{1}{1-x} P(x) = [x^M]\frac{x^S}{(1-x)^{S+N+1}} = \binom{M+N}{S+N}
                  $$
                  \vspace{0.3em}
            }
\end{frame}
\section{LOJ2409}
\begin{frame}[t]
\frametitle{LOJ 2409 - Sum} % https://vjudge.net/problem/LibreOJ-2409#author=GPT_en
\framesubtitle{Enunciado}
      \only<1-2>{
      Temos uma sequência $A$ de $N$ inteiros não negativos. Defina

      $$
      f_k = \sum_{i=1}^{N} A_i^k \mod 998244353
      $$

            Compute $f_1, f_2, \dots, f_n$.
      } 
      \only<2>{
      \begin{block}{Restrições}
           $1 \leq N \leq 2 \times 10^5$ \\
           $0 \leq A_i \leq 10^9$ 
      \end{block}
      }
\end{frame}
\begin{frame}[t]
\frametitle{LOJ 2409 - Sum} % https://vjudge.net/problem/LibreOJ-2409#author=GPT_en
\framesubtitle{Solução}
            \only<1>{
            Seja $F_i(x)$ a função geradora de $(a_n)_{n \in \mathbb{N}}$ com $a_n = A_i^n$. Veja que

            $$
            F_i(x) = \frac{1}{1-A_ix}
            $$

            Note que $P(x) = \sum_{i=1}^{N} F_i(x)$ é a função geradora de $(f_n)_{n \in \mathbb{N}}$.
            }
            \only<1->{
            Como computar $P(x)$?
            } 
            \only<2->{
                  Divisão e conquista!
            }
            \only<3->{
                  
                  \vspace{0.3em}

                  Por indução podemos mostrar que  a soma de $m$ $F_i$'s distintos, ou seja, $\sum_{i=1}^{m} F_{j_i}(x)$, é igual a $\frac{A(x)}{B(x)}$, onde $A, B$ são polinômios de grau $m$. 
            }
            \only<4->{

                  \vspace{0.3em}

                  Então para computar $\sum_{i=l}^{r}F_i(x)$ basta computar cada metade do intervalo recursivamente, e em seguida somar as frações calculadas usando $3$ convoluções: $\frac{A_1}{B_1}+\frac{A_2}{B_2} = \frac{A_1 B_2 + A_2 B_1}{B_1 B_2}$.

                  Obtendo $P(x) = \frac{C(x)}{D(x)}$ calculamos $D^{-1}(x)$ e convoluimos o resultado com $C(x)$ para obter a função geradora da resposta.
            }

\end{frame}

\section{ABC387G}
\begin{frame}[t]
     \frametitle{AtCoder ABC387G - Prime Circuit} %https://atcoder.jp/contests/abc387/tasks/abc387_g
     \framesubtitle{Enunciado}
      \only<1-2>{
                  Conte, módulo $998244353$, o número de grafos $G$ simples não direcionados e conexos de $N$ vértices satisfazendo a seguinte condição:
                  \begin{itemize}
                        \item O número de arestas em todo circuito$^{\dagger}$ de $G$ é um número primo.
                  \end{itemize} 

                  $\dagger$ Um circuito é uma trilha fechada (que começa e termina no mesmo vértice) que pode repetir vértices mas não arestas.
      } 
      \only<2>{
      \begin{block}{Restrições}
           $2 \leq N \leq 2.5 \times 10^5$ \\
      \end{block}
      }
\end{frame}
\begin{frame}[t]
     \frametitle{AtCoder ABC387G - Prime Circuit} %https://atcoder.jp/contests/abc387/tasks/abc387_g
     \framesubtitle{Solução}
            \only<1-3> {
            Vamos caracterizar quando $G$ cumpre a condição. Suponha que existem dois ciclos distintos $C_1, C_2 \in G$ que compartilham pelo menos um vértice, e que $C_1$ e $C_2$ têm um número primo de vértices. 
            \only<2-3> {
            \begin{block}{Lema:}
                  O grafo com arestas $E(C_1) \oplus E(C_2)$ tem um ciclo de tamanho par.
            \end{block}
            \only<3> {
            \begin{block}{Resumo da prova:}
            Mostre por indução em $|V(C_1)| + |V(C_2)|$. Os casos base são quando $C_1 \oplus C_2$ consiste de apenas um ciclo e quando $E(C_1) \cap E(C_2) = \emptyset$.
            \end{block}
            }
            }
            }

            \only<4> {
            Se não há interseção em vértices para todo par de ciclos distintos de $G$, então $G$ é um cacto e as únicas trilhas fechadas possíveis são os próprios ciclos. Então basta que todo ciclo tenha tamanho primo.
            }
            \only<5-6> {

            O problema se reduziu a contar o número de cactos com todos os ciclos tendo tamanho primo. Após comprimir todo ciclo em um "super vértice", o problema se torna uma generalização de contar árvores com $n$ vértices.            

            \vspace{0.5em}

            Sabemos contar quantas árvores de $n$ vértices existem?
            \only<6> {
                  Sim!
                  \begin{block}{Sequências de Prüfer}
                        Existe uma bijeção entre as sequências de $n-2$ inteiros variando de $1$ a $n$ e as árvores de $n$ vértices rotulados. Essa bijeção é construída usando as Sequências de Prüfer. 
                        
                  \end{block}
            }
            }
            \only<7-10>{
                  \only<7-8>{
                  Suponha que após a compressão de ciclos o grafo tenha $m$ super vértices e que eles tenham comprimido $s_1, \dots, s_m$ vértices do grafo original.}
                  \only<8-10> {Usando as sequências de Prüfer como inspiração, qual é o número de maneiras de conectar esses super vértices?
                  \only<9-10>{
            $$
                  n^{m-2} \prod_{i=1}^{m} s_i = \frac{1}{n^2} \prod_{i=1}^{m} ns_i
            $$
                  }
            
            }
            \only<10>{

                  Logo a resposta é igual a
                  
                  $$
                  \frac{1}{n^2} \sum_{\substack{s_1+\dots +s_m = n \\ s_i = 1 \text{ ou } 3 \leq s_i \in P}} \binom{n}{s_1, \dots, s_m} \prod_{i=1}^{m} ns_i C(s_i)
                  $$

                  

                  Onde $C(i)$ é o número de ciclos com $i$ vértices. Como temos um multinomial na conta, é útil pensar em EGFs.
            }
            }
            \only<11> {
                  Considere a EGF representando a contribuição de um único ciclo para o somatório:

                  $$
                        F(x) = \sum_{i \text{ primo}} niC(i)\frac{x^i}{i!} = nx + \sum_{i \text{ primo} \geq 3} \frac{n}{2} x^i
                  $$

                  Iterando pelo número $m$ de ciclos, a resposta será

                  $$
                  n![x^n]\sum_{m=0}^{\infty} \frac{F^m(x)}{m!} = n![x^n] e^{F(x)}
                  $$
            }
\end{frame}

%\section{ABC315Ex} 
\begin{frame}[t]
     \frametitle{AtCoder ABC315Ex - Typical Convolution Problem} %%https://atcoder.jp/contests/abc387/tasks/abc387_g
     \framesubtitle{Enunciado}
            Temos uma sequência $A$ de $N$ inteiros não negativos. Defina a sequência $(F_0, F_1, \dots, F_N)$ pelas fórmulas
            \begin{itemize}
            \item $F_0 = 1$
            \item $F_n = A_n \sum_{i+j<n}F_i F_j$ ($1 \leq n \leq N$)
            \end{itemize} 
            Encontre $F_1, \dots, F_N$ módulo $998244353$.
      \only<2>{
      \begin{block}{Restrições}
           $2 \leq N \leq 2 \times 10^5$ \\
           $0 \leq A_i < 998244353 $ \\
      \end{block}
      }
\end{frame}

\begin{frame}[t]
     \frametitle{AtCoder ABC315Ex - Typical Convolution Problem} %%https://atcoder.jp/contests/abc387/tasks/abc387_g
     \framesubtitle{Solução}
            \only<1-2> {
                  Neste problema não é possível convoluir diretamente, pois a princípio não conhecemos os termos intermediários $F_1, \dots F_{n}$. Para contornar isso, faremos uso de uma técnica chamada $\textsc{FFT Online}$.
                  
            }

            \only<2> {
                  A $\textsc{FFT Online}$ é capaz de resolver em $O(n \log^2 n)$ o seguinte problema:

                  Compute $c_1, \dots c_n$, com

                  $$
                        c_k = \sum_{i=0}^{k-1} a_i b_{k-1-i}
                  $$

                  onde $a_1, \dots, a_n$ e $b_1, \dots, b_n$ não são conhecidos, mas $a_k$ e $b_k$ podem ser computados após computarmos $c_k$.
                  
            }

            \only<3> {
                  Vamos agora esboçar o funcionamento da $\textsc{FFT Online}$:
                  \begin{itemize}
                        \item Usaremos Divisão e Conquista.
                        \item Para computar os termos em $[l, r]$:
                        \item Compute recursivamente os termos em $[l, m]$ onde $m = \frac{l+r}{2}$;
                        \item Compute a contribuição dos termos em $[l, m]$ para os termos em $[m+1, r]$;
                        \item Recursione para os termos em $[m+1, r]$.
                  \end{itemize}
            }

\end{frame}


\begin{frame}[fragile]
\frametitle{Pseudo-código - $\textsc{FFT Online}$}
\begin{lstlisting}
void solve(l, r) {
      if (l >= r) {
            // temos c_l, entao computamos a_l, b_l
            if (l == r) {..} 
            return; 
      }
      m = (l+r)/2
      solve(l, m)
      X = c[l,.., m]
      // ambos os termos em [l, r]
      P = convolution(X, X) 
      Y = c[0,..,min(r-l, l-1)]
      // exatamente um termo em [l, r]
      Q = convolution(X, Y) 
      // adiciona as contribuicoes em c[m+1, r]
      for (i em [m+1, r]) c[i] += P[ind1] + Q[ind2]
      solve(m+1, r)
}

\end{lstlisting}
\end{frame}
\begin{frame}

     \frametitle{AtCoder ABC315Ex - Typical Convolution Problem} %%https://atcoder.jp/contests/abc387/tasks/abc387_g
     \framesubtitle{Solução}
      Tendo o conhecimento da tecnologia necessária, o resto da solução fica como exercício para o leitor.
\end{frame}

\section{Sugestões}
\begin{frame}
      \frametitle{Problemas Sugeridos}
\begin{itemize}
\item Codar soluções para os problemas da apresentação!
\item \href{https://atcoder.jp/contests/abc226/tasks/abc226_h}{AtCoder ABC226H - Random Kth Max}
\item \href{https://atcoder.jp/contests/arc160/tasks/arc160_d}{AtCoder ARC160D - Mahjong}
\item \href{https://codeforces.com/gym/103743/problem/F
}{(CF Gym - Jiangsu 2022) F - Pockets}
\item \href{https://codeforces.com/gym/105949/problem/E}{(CF Gym - Sichuan 2025) E - Competition Graph}

\end{itemize}
\end{frame}



\end{document}
