% !TEX program = pdflatex
\documentclass[aspectratio=169]{beamer}
\usepackage[utf8]{inputenc}
\usepackage[T1]{fontenc}
\usepackage[brazil]{babel}
\usepackage{tikz}
\usepackage{lmodern}
\usetikzlibrary{fit}
\usetikzlibrary{arrows.meta,positioning}
\title{Da sequência de Prüfer para a árvore}
\subtitle{Exemplo passo a passo com TikZ}
\author{Você}
\date{\\Sequência de Prüfer escolhida: \( (4,\,4,\,6,\,6) \) com \(n=6\)}

% Cores e estilos rápidos
\definecolor{leaf}{RGB}{0,128,0}
\definecolor{seq}{RGB}{20,20,20}
\tikzset{
  vtx/.style={circle, draw, minimum size=7mm, inner sep=0pt, font=\footnotesize},
  used/.style={opacity=.25},
  edge/.style={Latex-Latex, thin}
}

\begin{document}

\begin{frame}
  \titlepage
\end{frame}

% ------------------------------------------------------------
% Layout dos vértices fixo para todas as etapas
\newcommand{\VertexLayout}{%
  % central
  \node[vtx] (4) at (0,0) {4};
  % folhas diretas de 4
  \node[vtx] (1) at (-1.5,1.5) {1};
  \node[vtx] (2) at (-1.5,-1.5) {2};
  \node[vtx] (6) at (2,0) {6};
  % folhas de 6
  \node[vtx] (3) at (3.5,1) {3};
  \node[vtx] (5) at (3.5,-1) {5};
}

% Macro para painel lateral com sequência restante e folhas
\newcommand{\SidePanel}[2]{%
  \begin{minipage}[t]{0.38\linewidth}
    \small
    \textbf{Sequência restante:}\\[2pt]
    {\Large\color{seq} #1}\\[6pt]
    \textbf{Menor folha disponível:}\\[2pt]
    {\large\color{leaf} #2}
  \end{minipage}%
}

\begin{frame}[t]
  \frametitle{Algoritmo (ideia)}
  \begin{itemize}
    \item A sequência de Prüfer para um rótulo \(1..n\) tem tamanho \(n-2\).
    \item Enquanto houver elementos na sequência: pegue a \alert{menor folha} (vértice de grau 1 não listado no prefixo restante) e conecte-a ao \alert{primeiro} número da sequência. Remova ambos.
    \item Ao final, restam dois vértices: conecte-os.
  \end{itemize}
\end{frame}

% ------------------------------------------------------------
% Etapas animadas
\begin{frame}[t]
  \frametitle{Exemplo passo a passo: sequência (4, 4, 6, 6)}
  \begin{columns}[T,onlytextwidth]
    \begin{column}{0.62\linewidth}
      % Área do desenho
      \begin{tikzpicture}[scale=1]
        \VertexLayout
        % Etapa 0: nada ligado ainda
        \only<1>{% estado inicial
          % destacar folhas iniciais: {1,2,3,5}
          \foreach \x in {1,2,3,5} {\node[draw=leaf, very thick, fit=(\x), minimum size=10mm, inner sep=0pt]{};}
        }
        % Etapa 1: liga 1--4
        \only<2->{\draw[edge] (1) -- (4);}
        % Etapa 2: liga 2--4
        \only<3->{\draw[edge] (2) -- (4);}
        % Etapa 3: liga 3--6
        \only<4->{\draw[edge] (3) -- (6);}
        % Etapa 4: liga 4--6
        \only<5->{\draw[edge] (4) -- (6);}
        % Etapa 5 (final): liga 5--6
        \only<6->{\draw[edge] (5) -- (6);}
      \end{tikzpicture}
    \end{column}
    \begin{column}{0.38\linewidth}
      % Painel lateral com sequência/folhas por etapa
      \only<1>{\SidePanel{$(\,\mathbf{4},\,\mathbf{4},\,\mathbf{6},\,\mathbf{6}\,)$}{1}}
      \only<2>{\SidePanel{$(\,\mathbf{4},\,\mathbf{6},\,\mathbf{6}\,)$}{2}}
      \only<3>{\SidePanel{$(\,\mathbf{6},\,\mathbf{6}\,)$}{3}}
      \only<4>{\SidePanel{$(\,\mathbf{6}\,)$}{4}}
      \only<5>{\SidePanel{$(\,)\,\text{(vazia)}$}{5}}
      \only<6>{\begin{minipage}[t]{\linewidth}\small
        \textbf{Fim da sequência.} Restam os vértices 5 e 6.\\[4pt]
        Conecte-os para completar a árvore.
      \end{minipage}}
    \end{column}
  \end{columns}

  \vspace{6pt}
  \begin{itemize}
    \item<1-> Folhas iniciais: $\{1,2,3,5\}$ (os que não aparecem na sequência).
    \item<2-> Passo 1: conecta $1$ a $4$; remove o primeiro $4$.
    \item<3-> Passo 2: conecta $2$ a $4$; remove o próximo $4$ (agora $4$ vira folha).
    \item<4-> Passo 3: conecta $3$ a $6$; remove o primeiro $6$.
    \item<5-> Passo 4: conecta $4$ a $6$; remove o último $6$.
    \item<6-> Passo 5: conecta os dois restantes $5$ e $6$.
  \end{itemize}
\end{frame}

% ------------------------------------------------------------
% Frame final com a árvore pronta e checagens
\begin{frame}[t]
  \frametitle{Árvore resultante e checagens}
  \begin{columns}[T,onlytextwidth]
    \begin{column}{0.62\linewidth}
      \begin{tikzpicture}[scale=1]
        \VertexLayout
\draw (1) -- (4);
\draw (2) -- (4);
\draw (6) -- (4);
\draw (3) -- (6);
\draw (5) -- (6);
      \end{tikzpicture}
    \end{column}
    \begin{column}{0.38\linewidth}
      \small
      \textbf{Grau esperado (contagem de Prüfer + 1):}
      \begin{itemize}
        \item $\deg(1)=1,\ \deg(2)=1,\ \deg(3)=1,\ \deg(5)=1$ (não aparecem)\\
        \item $\deg(4)=1+2=3$, \ $\deg(6)=1+2=3$ (cada um aparece 2 vezes)
      \end{itemize}
      \vspace{4pt}
      \textbf{Arestas}: $(1\!-\!4), (2\!-\!4), (3\!-\!6), (4\!-\!6), (5\!-\!6)$.
    \end{column}
  \end{columns}
\end{frame}

% ------------------------------------------------------------

\end{document}