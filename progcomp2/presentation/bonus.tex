\section{ABC315Ex} 
\begin{frame}[t]
     \frametitle{AtCoder ABC315Ex - Typical Convolution Problem} %%https://atcoder.jp/contests/abc387/tasks/abc387_g
     \framesubtitle{Enunciado}
            Temos uma sequência $A$ de $N$ inteiros não negativos. Defina a sequência $(F_0, F_1, \dots, F_N)$ pelas fórmulas
            \begin{itemize}
            \item $F_0 = 1$
            \item $F_n = A_n \sum_{i+j<n}F_i F_j$ ($1 \leq n \leq N$)
            \end{itemize} 
            Encontre $F_1, \dots, F_N$ módulo $998244353$.
      \only<2>{
      \begin{block}{Restrições}
           $2 \leq N \leq 2 \times 10^5$ \\
           $0 \leq A_i < 998244353 $ \\
      \end{block}
      }
\end{frame}

\begin{frame}[t]
     \frametitle{AtCoder ABC315Ex - Typical Convolution Problem} %%https://atcoder.jp/contests/abc387/tasks/abc387_g
     \framesubtitle{Solução}
            \only<1-2> {
                  Neste problema não é possível convoluir diretamente, pois a princípio não conhecemos os termos intermediários $F_1, \dots F_{n}$. Para contornar isso, faremos uso de uma técnica chamada $\textsc{FFT Online}$.
                  
            }

            \only<2> {
                  A $\textsc{FFT Online}$ é capaz de resolver em $O(n \log^2 n)$ o seguinte problema:

                  Compute $c_1, \dots c_n$, com

                  $$
                        c_k = \sum_{i=0}^{k-1} a_i b_{k-1-i}
                  $$

                  onde $a_1, \dots, a_n$ e $b_1, \dots, b_n$ não são conhecidos, mas $a_k$ e $b_k$ podem ser computados após computarmos $c_k$.
                  
            }

            \only<3> {
                  Vamos agora esboçar o funcionamento da $\textsc{FFT Online}$:
                  \begin{itemize}
                        \item Usaremos Divisão e Conquista.
                        \item Para computar os termos em $[l, r]$:
                        \item Compute recursivamente os termos em $[l, m]$ onde $m = \frac{l+r}{2}$;
                        \item Compute a contribuição dos termos em $[l, m]$ para os termos em $[m+1, r]$;
                        \item Recursione para os termos em $[m+1, r]$.
                  \end{itemize}
            }

\end{frame}


\begin{frame}[fragile]
\frametitle{Pseudo-código - $\textsc{FFT Online}$}
\begin{lstlisting}
void solve(l, r) {
      if (l >= r) {
            // temos c_l, entao computamos a_l, b_l
            if (l == r) {..} 
            return; 
      }
      m = (l+r)/2
      solve(l, m)
      X = c[l,.., m]
      // ambos os termos em [l, r]
      P = convolution(X, X) 
      Y = c[0,..,min(r-l, l-1)]
      // exatamente um termo em [l, r]
      Q = convolution(X, Y) 
      // adiciona as contribuicoes em c[m+1, r]
      for (i em [m+1, r]) c[i] += P[ind1] + Q[ind2]
      solve(m+1, r)
}

\end{lstlisting}
\end{frame}
\begin{frame}

     \frametitle{AtCoder ABC315Ex - Typical Convolution Problem} %%https://atcoder.jp/contests/abc387/tasks/abc387_g
     \framesubtitle{Solução}
      Tendo o conhecimento da tecnologia necessária, o resto da solução fica como exercício para o leitor.
\end{frame}